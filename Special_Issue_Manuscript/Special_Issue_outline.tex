\documentclass[11pt,a4paper]{article}
\usepackage{graphicx}
% uncomment according to your operating system:
% ------------------------------------------------
\usepackage[latin1]{inputenc}    %% european characters can be used (Windows, old Linux)
%\usepackage[utf8]{inputenc}     %% european characters can be used (Linux)
%\usepackage[applemac]{inputenc} %% european characters can be used (Mac OS)
% ------------------------------------------------
\usepackage{authblk}
\usepackage[superscript]{cite}
\usepackage[document]{ragged2e}
\usepackage[T1]{fontenc}   %% get hyphenation and accented letters right
\usepackage{mathptmx}      %% use fitting times fonts also in formulas
% do not change these lines:
\pagestyle{empty}                %% no page numbers!
\usepackage[left=35mm, right=35mm, top=15mm, bottom=20mm, noheadfoot]{geometry}
%% please don't change geometry settings!

\usepackage{fullpage}
\usepackage{amsfonts}
\usepackage{graphicx}
\usepackage{float}
\usepackage{amsmath}
\usepackage{chemfig}
\usepackage{indentfirst}
\usepackage{longtable}
\usepackage{array}
\usepackage{cellspace}
\usepackage{palatino}
%\usepackage{breqn}
\usepackage{amssymb}
\usepackage{verbatim}
\usepackage[colorlinks=true,citecolor=blue,linkcolor=blue]{hyperref}
\usepackage{siunitx}
\usepackage{xr}

% italicized boldface for math (e.g. vectors)
\newcommand{\bfv}[1]{{\mbox{\boldmath{$#1$}}}}
% non-italicized boldface for math (e.g. matrices)
\newcommand{\bfm}[1]{{\bf #1}}          

%\newcommand{\bfm}[1]{{\mbox{\boldmath{$#1$}}}}
%\newcommand{\bfm}[1]{{\bf #1}}
\newcommand{\expect}[1]{\left \langle #1 \right \rangle} % <.> for denoting expectations over realizations of an experiment or thermal averages

\newcommand{\var}[1]{{\mathrm var}{(#1)}}
\newcommand{\x}{\bfv{x}}
\newcommand{\y}{\bfv{y}}
\newcommand{\f}{\bfv{f}}

\newcommand{\hatf}{\hat{f}}

\newcommand{\bTheta}{\bfm{\Theta}}
\newcommand{\btheta}{\bfm{\theta}}
\newcommand{\bhatf}{\bfm{\hat{f}}}
\newcommand{\Cov}[1] {\mathrm{cov}\left( #1 \right)}
\newcommand{\T}{\mathrm{T}}                                % T used in matrix transpose

\newcommand\blfootnote[1]{%
	\begingroup
	\renewcommand\thefootnote{}\footnote{#1}%
	\addtocounter{footnote}{-1}%
	\endgroup
}


% begin the document
\begin{document}
	\thispagestyle{empty}
	%make title bold and 14 pt font (Latex default is non-bold, 16 pt)
	\title{\Large \textbf{Transferability of Mie n-6 force fields for predicting liquid shear viscosity at saturation and elevated pressures}}
	\author[1]{\large {\underline{Richard Messerly}}}%%[12 pt regular, presenting speaker underlined]
	
	\affil[1]{\textit{Thermodynamics Research Center (TRC), National Institute of Standards and Technology (NIST),
			Boulder, Colorado, 80305, USA}}
	
	\date{} % <--- leave date empty
	\maketitle\thispagestyle{empty} %% <-- you need this for the first page
	\begin{center}
		\title{\textbf{ABSTRACT}}\centering{}
	\end{center}
	\justify
	
\section*{Key points}

Mie and TAMie potentials are much better at saturation viscosities, despite not being fit directly to them
Viscosity density curve is much harder to reproduce
Viscosity pressure is adequately predicted with Potoff and TAMie
Branched alkanes have slightly worse performance

Propane is accurate to nearly 1 GPa
Butane agrees more closely with newer REFPROP correlation
C12 has similar results for Potoff and TraPPE?

Entropy scaling for isooctane?

Wrong torsional parameters for some isocompounds?

\section*{Outline}

\section{Introduction}

\begin{enumerate}
	\item Viscosity is an important property for designing chemical systems
	\item Viscosity data typically do not cover the entire range of $P \rho T$ of interest
	\item Prediction methods are typically quite poor for viscosity
	\item Molecular simulation is an attractive alternative, but two main challenges
	\begin{enumerate}
		\item Difficulty of obtaining reproducible results from simulation
		\item Unreliable force fields
	\end{enumerate}
    \item This manuscript applies the recent Best Practices to improve reproducibility such that it is possible to elucidate the difference in force fields
    \item Previous studies have suggested that UA models may be inadequate, while Gordon showed that a Mie potential could accomplish both VLE and viscosity
    \item This study tests whether the modern Mie potentials that are optimized for saturation thermodynamic properties are transferable to transport properties, e.g. shear viscosity
\end{enumerate}

\section{Methods}

\subsection{Force fields}

Copy the majority of this section from a previous publication

\begin{enumerate}
	\item United-atom and AUA models are the focus
	\item LJ 12-6 and Mie n-6
	\item Four force fields (although some have slightly different ethane parameters)
	\item None of these force fields specify bond types, so we used fixed bonds
	\item Torsions are the same for each force field
\end{enumerate}

\subsection{Simulation set-up}

\begin{enumerate}
	\item Two types of simulations performed, saturation and 293 K for compressed systems
	\item Saturation simulations use the REFPROP densities such that, in some cases, the force field is actually in a metastable state
	\item Performed some simulations at reported saturation conditions
	\item NPT performed for each replicate such that a distribution of box sizes is obtained
	\item Depending on the system, a simulation of 1, 2, 4, or 8 ns was used for the production stage
	\item Details are in supporting information
\end{enumerate}

\subsection{Data analysis}

Refer to Best Practices document

\begin{enumerate}
	\item Use 40\% sigma for cut-off
	\item Fit sigma to power model
	\item Fit viscosity to double exponential
	\item Bootstrap uncertainties by resampling replicate simulations
	\item 12 time origins
\end{enumerate}

We have tried to 

\section{Results}

\subsection{Saturated Liquid}

\subsubsection{n-Alkanes}

\begin{enumerate}
	\item Ethane is exception where Mie potential significantly over-predicts viscosity
	\item Propane, butane, n-octane all see significant improvement with Mie and TAMie
	\item C12 has spurious results
\end{enumerate}

Figures:

\begin{enumerate}
	\item Ethane
	\item C3, C4, C8
	\item C12, C16
\end{enumerate}

\subsubsection{Branched alkanes}

\begin{enumerate}
	\item Mie potential provides less improvement in these cases
\end{enumerate}

Figures:

\begin{enumerate}
	\item IC4, NEOC5
	\item IC5, IC8
	\item IC6, 23DMB, 3MP
\end{enumerate}

\subsection{High pressure fluid}

\subsubsection{n-Alkanes}

\begin{enumerate}
	\item Propane has accurate viscosity-P but not viscosity-rho
	\item Butane appears to agree more closely with recent REFPROP correlation
\end{enumerate}

Figures:

\begin{enumerate}
	\item Propane $\eta-\rho$ $\eta-P$
	\item Butane $\eta-\rho$ $\eta-P$
	\item n-Octane $\eta-\rho$ $\eta-P$
%	\item n-Dodecane $\eta-\rho$ $\eta-P$?
\end{enumerate}

\subsubsection{Branched alkanes}

\begin{enumerate}
	\item Similar to n-alkanes? 
	\item Wrong torsions matters?
\end{enumerate}

Figures:

\begin{enumerate}
	\item Isobutane $\eta-\rho$ $\eta-P$
	\item Isopentane $\eta-\rho$ $\eta-P$
%	\item Isohexane $\eta-\rho$ $\eta-P$?
	\item Isooctane $\eta-\rho$ $\eta-P$
%	\item Neopentane $\eta-\rho$ $\eta-P$?
	\item 3-methylpentane $\eta-\rho$ $\eta-P$?
%	\item 2,3-dimethylbutane $\eta-\rho$ $\eta-P$?
\end{enumerate}

\section{Discussion/Limitations}

\begin{enumerate}
	\item Discussion
	\begin{enumerate}
		\item Mie potentials parameterized with VLE data provide significant improvement over LJ 12-6
		\item Potoff over-predicts $\eta-\rho$ dependence while TAMie is fairly accurate
		\item Potoff appears to be slightly more accurate for $\eta-P$
		\item Branched alkanes are not as accurate, perhaps assumption of transferability or torsional parameters
	\end{enumerate}
    \item Limitations
    \begin{enumerate}
    	\item Largest viscosity simulations are slow to converge and unclear if simulations are sufficiently long
    	\item Tail-corrections could impact dynamics
    	\item Using REFPROP saturation conditions instead of force fields
    \end{enumerate}
\end{enumerate}

\section{Conclusions}

\section{Acknowledgments}

\section{Supporting Information}

\subsection{Gromacs input files}

\begin{enumerate}
	\item Include all the .gro files
	\item Include all the .top file templates
	\item Include .mdp files
	\item Or we can just include an example and then refer them to the GitHub website
\end{enumerate}

\subsection{Tabulated values}

\begin{enumerate}
	\item Ethane
	\begin{enumerate}
		\item Saturation
        \begin{enumerate}
        	\item Potoff
        	\item TraPPE
        	\item AUA4
        	\item TAMie
        \end{enumerate}
		\item T293 highP
		\begin{enumerate}
			\item Potoff
			\item TraPPE
			\item AUA4
			\item TAMie
		\end{enumerate}
	\end{enumerate}
    \item Propane
   	\begin{enumerate}
    	\item Saturation
    	\begin{enumerate}
    		\item Potoff
    		\item TraPPE
    		\item AUA4
    		\item TAMie
    	\end{enumerate}
    	\item T293 highP
    	\begin{enumerate}
    		\item Potoff
    		\item TraPPE
    		\item AUA4
    		\item TAMie
    	\end{enumerate}
    \end{enumerate}
    \item n-Butane
    \begin{enumerate}
    	\item Saturation
    	\begin{enumerate}
    		\item Potoff
    		\item TraPPE
    		\item AUA4
    		\item TAMie
    	\end{enumerate}
    	\item T293 highP
    	\begin{enumerate}
    		\item Potoff
    		\item TraPPE
    		\item AUA4
    		\item TAMie
    	\end{enumerate}
    \end{enumerate}
    Repeat for all other compounds with corresponding potentials    
\end{enumerate}

\subsection{Finite-size effects}

\begin{enumerate}
	\item Simulation results for 100, 200, 400, and 800 molecules
\end{enumerate}

\subsection{Simulation length effects}

\begin{enumerate}
	\item Verified that 1 ns is long enough for larger compounds
\end{enumerate}

\subsection{Validation Runs}

\begin{enumerate}
	\item Ethane NIST
	\item n-Octane Literature
\end{enumerate}

\subsection{Bond types, Harmonic vs LINCS}

\begin{enumerate}
	\item Propane and n-butane with harmonic (arbirary bond constant) shows systematic increase
\end{enumerate}

\subsection{Green-Kubo analysis}

\begin{enumerate}
	\item Raw data, i.e., multiple replicates with the average
	\item Exclude low time data and have a heurestic for determining the cut-off time
\end{enumerate}

Example analysis, i.e., bootstrap distribution, replicates

\subsection{MCMC?}

\end{document}
